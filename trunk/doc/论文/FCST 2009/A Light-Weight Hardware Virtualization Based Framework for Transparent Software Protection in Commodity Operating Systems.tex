\documentclass[conference]{IEEEtran}
\usepackage[pdftex]{graphicx}
\usepackage{amsmath}
\usepackage[tight,footnotesize]{subfigure}
\usepackage{cite}
\usepackage{booktabs}
\usepackage{listings}
\lstset{language=C++,linewidth=0.4\textwidth,
basicstyle=\ttfamily\small, commentstyle=\small\it}


% *** Do not adjust lengths that control margins, column widths, etc. ***
% *** Do not use packages that alter fonts (such as pslatex).         ***
% There should be no need to do such things with IEEEtran.cls V1.6 and later.
% (Unless specifically asked to do so by the journal or conference you plan
% to submit to, of course. )


% correct bad hyphenation here
\hyphenation{op-tical net-works semi-conduc-tor}


\begin{document}
%
% paper title
% can use linebreaks \\ within to get better formatting as desired
\title{HBSP: A Lightweight Hardware Virtualization Based Framework for Transparent Software Protection in Commodity Operating Systems}


% author names and affiliations
% use a multiple column layout for up to three different
% affiliations
\iffalse
\author{
\IEEEauthorblockN{Miao Yu}
\IEEEauthorblockA{School of Software,\\Shanghai Jiaotong University\\
Email: yumiao@sjtu.edu.cn}
\and
\IEEEauthorblockN{Peijie Yu}
\IEEEauthorblockA{School of Software,\\Shanghai Jiao Tong University\\
Email: yupeijie@sjtu.edu.cn} \and \IEEEauthorblockN{Shang Gao}
\IEEEauthorblockA{School of Software,\\Shanghai Jiaotong
University\\Email: gaoshang@sjtu.edu.cn}}
\fi

\author{Miao Yu, Peijie Yu, Shang Gao, Qian Lin, Min Zhu, Zhengwei
Qi\\ \\
School of Software, Shanghai Jiao Tong University\\
\{superymk,yupiwang,chillygs,linqian,zhumin,qizhwei\}@sjtu.edu.cn}


\maketitle

\begin{abstract}
%\boldmath
Commodity operating systems are usually large and complex, leading
host-based security tools often fail to provide adequate
protection from malware. The execution environment for software is
untrusted. As a result, most software currently uses various ways
to defend malware attack. However, those approaches not only raise
the complexity of the software but also fail to offer an engrained
security solution. The focal point in the software protection
battle is how to protect effectively versus how to conceal the
protector from untrusted OS. This paper describes a lightweight,
transparent and flexible architecture framework called HBSP
(Hypervisor Based Software Protector) for software protection.
HBSP, which is based on hardware virtualization extension
technology such as Intel VT, and by taking advantage of
Memory-Hiding strategy, resides completely outside of the target
OS environment.  Our security analysis and the performance
experiment results based on SPEC 2006 demonstrate that HBSP
effectively protects applications running on unmodified Windows
XP, while the total overhead to existing application is only
0.29\%.

\bigskip
%\noindent\textfb{Keywords:} keywords1,keywords2
%\IEEEoverridecommandlockouts

\begin{IEEEkeywords}
Hardware Virtualization, Lightweight Transparent Software
Protection, Commodity Operating Systems, Memory-Hiding, HBSP
\end{IEEEkeywords}

\end{abstract}

\IEEEpeerreviewmaketitle

\bigskip

\section{Introduction}


Nowadays, commercial operating systems are deployed every corner
in the home, office, and government, manage various commercial
software on them. Unfortunately, most of the operating systems
cannot provide adequate security and software protection due to
the design and hardware limitation. Consequently, though it will
raise the total cost on software development, nearly all the
commercial software needs to implement its own additional
protection module.

Current approaches of software protection can be separated into
two categories \cite{IEEEhowto:1}. One is to implement both static
and dynamic code validation through the insertion of objects into
the generated executable, such as watermarking and software
birthmark [2, 3] which utilize inherent characteristics of a
program to identify it. Another more radical method is to protect
software with hardware supporting \cite{IEEEhowto:20}. The target
program is divided into various parts which can be run in an
encrypted form on secure coprocessors. Smart card can be viewed as
one of the types which could store sensitive computations and data
but offer no direct I/O to users \cite{IEEEhowto:4}.

However, the risk of attacking the target software still exists
for the following reasons: first, hardware architecture protection
is limited. Hardware architecture defines that the code running in
the privileged mode owns system-wide access to the resources,
while the code in the user mode can only access a limit
range\cite{IEEEhowto:21}. So once the malicious code or analyzing
tools are running in the privileged mode, no more powerful mode
can be used to stop it.

Second, there are always bugs and debugging functions in the
commodity operating systems. The kernel and the $\text3^\text{rd}$
party device drivers of a commodity operating system contain
millions of lines of code\cite{IEEEhowto:7}. Meanwhile, nearly all
the commercial operating systems provide tools to see other
processes' address space and attach a thread to each process for
debug use. As a result, there is no way to stop someone who intends
to crack commercial software.

The growing popularity of hardware virtualization motivates our
new solution of software protection. Previous efforts address to
retrofit a trusted execution environment on commodity system by
separating malicious code and system kernel in isolated VMs [12,
13, 14], or using active monitor to handle security sensitive
behaviors [15, 17]. However, these approaches pose a substantial
barrier to adoption as their low performance or critical
requirement of modification to application code or system kernel.

In this paper, we demonstrate a \emph{transparent, external}
approach to software protection. Unlike the priors, our design
requires no code modification of the existing operating systems or
adding additional hardware. Even the control flow between the
application and the OS kernel remains unchanged. HBSP makes use of
Intel VT to create a transparent execution environment instead of
employing in-guest components or partial/full system
virtualization, for none of these approaches satisfies all
transparency requirements. Along with Intel VT, Memory-Hiding
technology makes it feasible for hypervisor to conceal itself in a
private memory region. As a Consequence, the hypervisor is able to
run transparently under the commercial operating systems while
keeping the ability of monitoring the target application's
execution and state.

Our work represents the following contributions:
\begin{itemize}
\item A lightweight framework with least temporal and spatial
overhead of the communication between the guest OS and hypervisor
as our experiment results revealed. \item Implementation of a
transparent memory-protecting mechanism which takes advantage of
hardware virtualization and Memory-Hiding technology offering
protection to hypervisor by memory remapping. \item Description of
the flexibility and extensibility of HBSP which offers a rich set
of interfaces for configuration as well as being compatible with
other hardware virtualization platform.
\end{itemize}

The following section presents the design goals of HBSP. In
section 3, we explain the framework implementation and the
challenges when introducing in Memory-Hiding. Section 4 describes
an example to protect software with the help of HBSP framework in
detail, as well as how it can prevent cracking by dynamic
analysis. Section 5 shows how we apply our implementation to a
default installation of the Windows XP and evaluates the overall
performance. Section 6 reviews related work. Finally we will
conclude the ideas in section 7.

\bigskip

\section{Design Goals}
Hardware Enabled Virtualization (HEV) Technology can be adopted as a
way of software defense [10, 11, 17]. Our motivation is to offer
another line of software protection even on an untrusted environment
while keeping external and transparent to existing software, as well
as being adopted easily. As a result, we build the lightweight HBSP
with the following advantages:

{\bf\emph {Install/Uninstall on the fly}} \hskip 1em plus 0.5em
minus 0.4em\relax HBSP's key idea is to load the host OS into a
virtual machine at runtime. By install/uninstalling hypervisor on
the fly, it is much easier to use and debug the hypervisor, while
not affecting the legacy OS and application's integrity, as well as
the user experience.


{\bf\emph {Flexible Configuration}} \hskip 1em plus 0.5em minus
0.4em\relax In order to support various ways to protect software,
HBSP supplies a rich set of interfaces to configure VMCS (Virtual
Machine Control Structure) struct and memory strategy. Hypervisor
can be configured at both the compile time and the runtime. However,
it's not advised to keep all sensitive content in hypervisor for
performance reasons.

{\bf\emph {Support for other HEV technology}} \hskip 1em plus
0.5em minus 0.4em\relax Currently HBSP supports Intel VT
technology. Other technologies such as AMD SVM are also largely
adopted in the market. To minimize the cost when refactoring the
hypervisor to support these platforms, we build a layer to hide
the hardware details. Hence, extending hypervisor platform support
will not affect other parts, as well as the protected software.

\bigskip

\section{system implementation}
It is challenging to implement the HBSP running on unmodified
commodity operating systems. In this section, we firstly describe
the HBSP architecture, and then introduce its control flow and how
it controls the guest OS. The analysis and implementation of
Memory-Hiding technology will be discussed at the end of this
section.

\subsection{HBSP Architecture}
\begin{figure}[!htb]
\begin{center}
\includegraphics[width=0.45\textwidth]{fig1.png}
\end{center}
\caption{{\bf HBSP Architecture} \hskip 1em plus 0.5em minus
0.4em\relax HBSP consists of three layers. User defined hypervisor
will be placed in the top layer, with the help of the HBSP
Services and the HBSP Interfaces. } \label{Figure 1.}
\end{figure}

Figure 1 presents the HBSP architecture. Tailored from BluePill
Project \cite{IEEEhowto:5} and migrated on to the x86 platform to
expand its usage scope, HBSP is divided into three layers: Platform
Related Layer, HBSP Interfaces and $\text3^\text{rd}$ Hypervisor
Layer.

{\bf\emph{Platform Related Layer}} \hskip 1em plus 0.5em minus
0.4em\relax   This layer is used to mask implementation differences
among HEV technologies (e.g. HEV-related instructions, checking
platform, VMCS configurations, etc) and provide unique interfaces to
upper layers.

In Platform Related Layer, HBSP requires implementation of HEV
technology supporting the following routines (Since HBSP is tailored
from BluePill Project, we keep most of the procedure names and some
function implementations), listed in Table 1.


\iffalse
\begin{itemize}
\item {ArchIsHvmImplemented() \hskip 1em plus 0.5em minus
0.4em\relax This procedure checks if a hardware platform supports
HEV technology.} \item {ArchInitialize() \hskip 1em plus 0.5em minus
0.4em\relax This procedure takes the responsibility of initializing
the hypervisor and guest machine, it can also enter the VMM if
needed.} \item {ArchVirtualize() \hskip 1em plus 0.5em minus
0.4em\relax This routine is responsible for starting the guest
machine. As a result, the original OS is put into the virtual
machine and continue executing instructions with no sense to the
underlying hypervisor.} \item {ArchDispatchEvent() \hskip 1em plus
0.5em minus 0.4em\relax This function dispatches events to the
proper handler. It is invoked under the following condition: the
hypervisor is using the Default Event Handler supplied in HBSP and a
\#VMEXIT event is occurred in virtual machine.}
\end{itemize}
\fi

\begin{table}[!t]
\caption{{\bf \hskip 2em plus 1.5em minus 0.4em\relax The Required
Routines in Platform Related Layer.}} \label{table_10.} \centering
\begin{tabular}{c|c}
\toprule
\makebox[2.7cm]{ArchIsHvmImplemented()}  &
\parbox[c]{5.0cm}{This procedure checks if a
hardware platform supports HEV technology.}\\
\midrule
\renewcommand{\arraystretch}{4.5}
ArchInitialize() & \parbox[c]{5.0cm}{This procedure takes the responsibility of initializing the hypervisor and guest machine, it can also enter the VMM if needed.} \\
\midrule
\renewcommand{\arraystretch}{5.5}
ArchVirtualize() & \parbox[c]{5.0cm}{This routine is responsible for starting the guest machine. As a result, the original OS is put into the virtual machine and continue executing instructions with no sense to the underlying hypervisor.}\\
\midrule
\renewcommand{\arraystretch}{6.5}
ArchDispatchEvent() & \parbox[c]{5.0cm}{This function dispatches events to the proper handler. It is invoked under the following condition: the hypervisor is using the Default Event Handler supplied in HBSP, and a \#VMEXIT event is occurred in virtual machine.}\\
\bottomrule
\end{tabular}
\end{table}



{\bf\emph{HBSP Interfaces}} \hskip 1em plus 0.5em minus 0.4em\relax
HBSP exports a set of interfaces called \emph{Strategy} to meet the
requirement on the framework's behavior and resource management from
hypervisor developer. For instance, with respect to Memory Strategy
Interfaces, in order to allocate memory for future use, the Platform
Related Layer calls proper functions declared in this set of
interfaces. If Memory Hiding Strategy is used as the definition of
the current memory strategy, HBSP always uses its private page table
to allocate memory for the caller. While using the Default Memory
Strategy results in all the memory management tasks will be
delegated to Windows system.

{\bf\emph{$\text3^\text{rd}$ Hypervisor Layer}} \hskip 1em plus
0.5em minus 0.4em\relax This layer is concentrated on implementing
customized hypervisor logic. Building on top of HBSP Interfaces, it
supplies services and default HBSP Interfaces implementation to
accelerate constructing a hypervisor. By default, the build-in HBSP
Services mainly include:

\begin{itemize}
\item Default Memory Management \hskip 1em plus 0.5em minus
0.4em\relax HBSP supports two memory management strategies by
default. The Default Memory Strategy uses the Windows kernel API to
manage the hypervisor's code and data memory. Thus, any allocation
and deallocation to this address space can be detected by other
processes. In contrast, the Memory-Hiding Strategy is dedicated to
keep the hypervisor out of the view of guest OS; see Section 3.3 for
details.

\item Default Event Handler \hskip 1em plus 0.5em minus
0.4em\relax Default Event Handler is used to register a callback
function with the indicated \#VMEXIT Reason. When an \#VMEXIT event
happens, hardware automatically records the \#VMEXIT Reason in the
VMCS before executing the first instruction in hypervisor. For the
performance optimization, each \#VMEXIT Reason is linked to an
independent event handler chain in HBSP. Thus it cuts down the total
time spending in the hypervisor for limiting the length of the
chain.
\end{itemize}

Currently, HBSP is designed to support only one guest machine at the
moment, though a typical hypervisor can support more virtual
machines as a nature. Regardless the guest VM amount, HBSP offers a
new approach in protecting software. It brings in a valuable layer
of protection, and requires no change to the hardware and the
operating system.

\subsection{HBSP Control Flow}
Since HBSP is based on HEV technology, it plays a role as guest
machine controller between the guest machines and the physical
hardware. Considering a single guest machine running on the top
layer, at any time, the whole system can be in only one context
among the following three types: guest application (Ring 3), guest
kernel (Ring 0/1) and hypervisor (VMM). Both the hypervisor
interested events and the ones which should not be handled in the
guest machine will be transferred to hypervisor and activate it.
Later, hypervisor transfers control back to the guest machine
explicitly after it finishes handling the events.

\begin{figure}[!htb]
\begin{center}
\includegraphics[width=0.4\textwidth]{fig2.png}
\end{center}
\caption{{\bf Basic State Transition Diagram}} \label{Figure 2.}
\end{figure}

As shown in Figure 2, both guest user mode instructions and guest
kernel mode instructions have the capability to trigger the hardware
to generate a \#VMEXIT event then transferring control to the
hypervisor. For example, when a guest application executes the CPUID
instruction, a \#VMEXIT event is generated and hardware activates
the hypervisor automatically to handle the event (Transition 1). The
hypervisor examines the stored guest machine state to determine how
to handle it properly, then uses VMX instructions to force the
physical hardware to resume the guest machine's execution and
transfer the control back to the guest machine (Transition 2).

Guest kernel has more chances to trap into the hypervisor. For
instance, a guest machine always triggers \#VMEXIT event once to
read MSR registers if and only if the hypervisor is configured to
monitoring RDMSR instructions with the help of HBSP. Then the
hypervisor does the same handling process as that with application
(Transition 3, 4).

HBSP also supports intercepting the guest machine on accessing
physical resources such as I/O related instructions. A hypervisor
mediating between guest machine and physical hardware can always
perform additional operations on the I/O accesses (Transition 5, 6)
before resuming to the guest machine. With such approach, a
hypervisor can cheat a malicious kernel and software to protect
applications.

\subsection{Memory-Hiding Technology}
A hypervisor is vulnerable if it can be accessed from the guest OS.
To improve the approach of hypervisor based software protection,
Memory-Hiding technology is applied to conceal the hypervisor
completely. In this section, we firstly describe the transparency
limitation without Memory-Hiding, and then analyze its
implementation. Finally, we will show the effectiveness after the
hypervisor is turned on.

A typical commercial OS needs to build and manage process page
tables for address translation. Consequently, the mapping from the
hypervisor's virtual address  (VA) to real physical address
(PA$_\text{real}$) is created as {P(VA, PA$_\text{real}$) in the
system page table. Being accessible from the guest OS, a hypervisor
can be easily invalidated in the face of a malicious kernel. Figure
3 shows the limitation under this situation.

\begin{figure}[!htb]
\begin{center}
\includegraphics[width=0.4\textwidth]{fig3.png}
\end{center}
\caption{{\bf Translation without Memory-Hiding.} \hskip 1em plus
0.5em minus 0.4em\relax The space of hypervisor can be accessed
even if the hypervisor itself is running under the guest machine.}
\label{Figure 3.}
\end{figure}


Memory-Hiding technology patches the indicated PTE in the guest
OS's page table. It clones the current page table for private
usage, and then changes the mapping from P(VA, PA$_\text{real}$)
in the hypervisor own address space to $\text{P}^{'}$(VA,
PA$_\text{spare}$), where PA$_\text{spare}$ refers to the physical
address of a special spare page. This strategy makes all access to
the hypervisor be swept out from the patched page table. When the
execution context switches to hypervisor, the private page table
takes effect and the hypervisor can reference itself. As shown in
Figure 4, once leaving the hypervisor context, the patched page
table with mapping of $\text{P}^{'}$(VA, PA$_\text{spare}$) will
be enabled automatically by hardware, making the hypervisor
obscure again.

\begin{figure}[!htb]
\begin{center}
\includegraphics[width=0.45\textwidth]{fig4.png}
\end{center}
\caption{{\bf Translation Procedure with Memory-Hiding.}}
\label{Figure 4.}
\end{figure}

We make no attempt to let the Memory-Hiding technology support PAE
mode in current system, though it is a promotional functionality
and not hard to implement.

\bigskip

\section{Case Study: Protecting Software with HBSP}
In order to verify the efficient and effectiveness of HBSP, we
develop a simple hypervisor called \emph{SNProtector} to store the
application's serial key validation algorithm and registration
state. Our protection goal is: even the source code of the
application field is publicly known, the application still remains
protected as long as the SNProtector hypervisor is active.

\begin{figure}[!htb]
\begin{center}
\includegraphics[width=0.45\textwidth]{fig5.jpg}
\end{center}
\caption{{\bf SNProtector Design and Usage Model.}} \label{Figure
5.}
\end{figure}

Figure 5 shows the design and usage model of SNProtector. To fulfill
the protection goal, it is important to use Memory-Hiding Strategy
to conceal its code and data segment and render guest OS
imperceptible. Thus, it is hard for an attacker to find out and lock
the registration state in the hypervisor space in that the hidden
pages are never referenced in the guest OS's virtual space.

It's also vital to realize that the conventional way of protecting
software using serial key is vulnerable because it merely verifies
the serial key for only once. To address this problem, SNProtector
sets up a timer simulator, which will be triggered to verify the
registration state in the hypervisor field after a fixed amount of
CR3 switches. A better choice of adopting VMX Preemption Timer is
omitted here due to the limited hardware support currently.
Nevertheless, the design reliably renders attackers impossible to
find out which instruction exactly causes the timer trap into the
SNProtector hypervisor and intercept it on a multi-task operating
system like Windows.

\begin{figure}
\begin{lstlisting}
main:
  // if reqire unload hypervisor
  // Reveal hypervisor then exit
  if( reqRevealHypervisor ) {
      RevealHypervisor();
      exit;
  }

  ReadIn(&UserName,&SerialNumber);

  // Hide hypervisor
  // Pass the reg info into hypervisor
  HideHypervisor();
  bRegState = VerifySN(&UserName,
    &SerialNumber);

  // I am Cracker!!!
  // bRegState = TRUE;

  // Output proper information
  // in the client side.
  if( bRegState ) {
      RegSuccessful();
  } else {
      RegFailure();
  }

  // To simulate a real commercial software
  while( bRegState ) {
      printf("WorkWork!\n");
      Sleep(2000);
  }

  exit;
\end{lstlisting}
\caption{{\bf The Protected Software Implementation}}
\label{Figure 10.}
\end{figure}

Figure 6 demonstrates a typical implementation of protected program.
Uncommenting the line \emph{"bRegState = TRUE;"} will crack and lock
the registration state in the application field, but the hypervisor
field. Consequently, the tamper behavior is detected immediately as
long as the SNProtector hypervisor is running in the background. So
our approach is effective even in the condition that application's
source code is public.

Taking performance optimization into account, SNProtector also
stores the registration state in the application field to reduces
the clock cycles at runtime as the application doesn't need to
query the registration state initiatively by means of triggering
traps. Although it costs tens of hundreds of cycles to handle a
VMM trap, the overall performance overhead when applying the
protection is limited.

\bigskip

\section{Experiments and Results}
All experiments were conducted on a desktop computer with a
1.83GHz Intel Core2 Duo processor and 2GB RAM. SNProtector is
installed on both the processor cores. Windows XP SP3 is selected
as the guest operating system of the testbed.

\begin{table}[!t]
%\renewcommand{\arraystretch}{1.0}
\caption{{\bf \hskip 2em plus 1.5em minus 0.4em\relax
Microbenchmarks.} \textnormal{\hskip 1em plus 1.5em minus
0.4em\relax Clock cycles of execution CPUID \hskip 8em plus 1.5em
minus 0.4em\relax instruction before and after installing
SNProtector. }} \label{table_1.} \centering
\begin{tabular}{|c|c|c|}
\hline \  & \makebox[2.4cm]{Before Loading}  & \makebox[2.4cm]{After Loading}\\
\ & SNProtector & SNProtector \\
\hline
Execution Cycle & 268 & 2660\\
\hline
\end{tabular}
\end{table}

\begin{figure*}
\begin{center}
\includegraphics[width=0.77\textwidth]{fig7.png}
\end{center}
\caption{{\bf SPEC CINT 2006 Benchmarks.}} \label{Figure 6.}
\end{figure*}

\begin{figure*}[!htb]
\begin{center}
\includegraphics[width=0.77\textwidth]{fig8.png}
\end{center}
\caption{{\bf SPEC CFP 2006 Benchmarks.}} \label{Figure 7.}
\end{figure*}

\emph{Microbenchmarks} \hskip 1em plus 0.5em minus 0.4em\relax
Table 2 highlights the results of microbenchmarks that measure the
overhead of intercepting instruction execution by SNProtector. In
this experiment, the benchmarks exhibited low performance on
executing intercepted instructions on guest machine, nearly 9
times more cycles needed to handle the interception after loading
SNProtector. The reason is that trapping to hypervisor introduces
in overhead due to access VMCS region, so does invoking the proper
callback function.

\emph{Application Benchmarks} \hskip 1em plus 0.5em minus
0.4em\relax Though the microbenchmarks show an unacceptable
result, the performance impact on the real application is
imperceptible. Figure 7 and 8 present results from the SPEC
CPU2006 integer suite and float point suite, illustrating that
running the hypervisor only brings in a little overhead.

In these tables, program run time is measured to scale the
performance. When the individual benchmarks are considered, only
GCC has a little higher overhead. This derives from GCC's
relatively high system call rates, thus accessing CR3 register
more often than other programs, which always causes trapping into
hypervisor.

The web server experiment, measuring the throughput bytes per
second (bps), used the default configuration of APACHE 2.2.11
win32 version. A test website is created with 10 random files, the
size of which varies from 1KB to 8MB. The http\_load tool was set
up on a remote host to generate requests for fetching all of these
files with 100 concurrent connections. The client and server were
connected by a 100Mbps switch. The overhead of running the
SNProtector is 0.55\%. Combined with the overall SPEC benchmarks,
as shown in Figure 9, the total overhead to the guest machine is
0.29\% on average.


\begin{figure}[!htb]
\centering
\includegraphics[width=0.45\textwidth]{fig9.png}
\caption{{\bf Application Benchmark Summary.}} \label{Figure 8.}
\end{figure}


\bigskip

\section{related work}
HEV Technology enables transparent intercepting guest OS exceptions
and interrupts. This has been leveraged in the New Blue Pill project
to intercept external timer source like HPET and Real Time
Clock\cite{IEEEhowto:8}. Also, it enables another step of
translating guest physical address into real machine physical
address. Systems which employ this ability can be used in
virtualizing physical address\cite{IEEEhowto:9} and transparent page
sharing\cite{IEEEhowto:10} , as well as transparent VM migration
among physical machines \cite{IEEEhowto:11}. Our approach modifies
the guest OS's page table directly to provide different views of
memory to guest machine and hypervisor.

Another example is Overshadow\cite{IEEEhowto:6}, which also can be
implemented by HBSP. With the help of additional layer of address
translation, it provides different views of the sensitive
application's address space according to the current context.
Without affecting the existing OS and legacy protected
application, even the hardware, kernel and other programs can only
get the encrypted content from the protected application's virtual
space. Considering only instruction interception at the moment,
the performance of SNProtector is much higher than Overshadow
comparatively.

Many previous systems have attempted to provide a higher-assurance
execution environment by means of building separate VMs.
Proxos\cite{IEEEhowto:12} runs its protected applications in
trusted VMs, with the ability of using the resource in the
untrusted VM. Similarly, iKernel\cite{IEEEhowto:13} improves the
commodity operating system's reliability and security by isolating
the buggy and malicious device drivers running on separate virtual
machine. To establish a configurable trust between applications
and operating systems, partition system call interface  or
separation of privilege\cite{IEEEhowto:15} is utilized to enable
secure level code running in different VMs. Hypervisor based
monitoring on malicious behaviors [15, 17] is used to handle
certain security sensitive instructions. Focusing on the similar
design goal, our approach excels at higher performance and less
affect to legacy OS kernel due to the optimized design. Proxos
\cite{IEEEhowto:12} requires code modification to both application
and the kernel. While Igor Burdonov's work [14], Ether [15] and
Lares [17] pose a high performance penalty and iKernel
\cite{IEEEhowto:13} hasn't range its experiment statistics on
temporal overhead test.

Intel's Trusted Execution Technology\cite{IEEEhowto:17} is another
hardware-based software security approach, providing isolated
protected execution environment, which offers no privilege to
unauthorized software even to observe. Furthermore, it provides
protected input and storage channel to ensure the data security.
This approach can be regarded as a complement of our Memory-Hiding
technology, albeit only available on some special hardware.

\bigskip

\section{conclusion}
In this paper, we have presented the architecture and design of
HBSP, which can be used to implement external hypervisor running
transparently under the guest OS and intercepting indicated guest
machine actions without modifying existing OS and hardware, even
software. Especially, the hypervisor constructed by HBSP can be
install/uninstalled on the fly and hides its space from the OS
view with the help of Memory-Hiding Strategy.

Memory-Hiding is entitled by patching the page table of guest OS
while holding a real one in the hypervisor. This causes some
anti-debug tools and hypervisor detect tools invalid in the face
of HBSP. Thus it constructs a safer execution environment for a
hypervisor layer application or software protector to store
sensitive data and codes.

As a prototype implementation, we build a simple serial key
protector using HBSP. A series of experiments on Windows OS have
proven that our approach introduces little overhead to the existed
environment. Still, a series of interesting research opportunities
and topics remain for the future study, such as pre-OS Hypervisor
Loading and enabling Self-Verification based on Intel TXT
technology. We believe that HBSP is a practical approach to protect
software in commodity operating systems.
\bigskip

\section*{Acknowledgment}
This work is supported by National Natural Science Foundation of
China (Grant No.60773093 and 60873209), the Key Program for Basic
Research of Shanghai (Grant No.09JC1407900), Microsoft Research Asia
Young Teacher Funding.

\bigskip

\begin{thebibliography}{99}

\bibitem{IEEEhowto:1}
Zambreno, J.; Honbo, D.; Choudhary, A.; Simha, R.; Narahari, B..
 \emph {High-Performance Software Protection Using Reconfigurable
Architectures.}\hskip 1em plus 0.5em minus 0.4em\relax Proceedings
of the IEEE Volume 94, Issue 2, Feb. 2006 Page(s):419 - 431.

\bibitem{IEEEhowto:2}
Xiaoming Zhou; Xingming Sun; Guang Sun; Ying Yang; \emph {A
Combined Static and Dynamic Software Birthmark Based on Component
Dependence Graph.}\hskip 1em plus 0.5em minus 0.4em\relax
Intelligent Information Hiding and Multimedia Signal Processing,
2008. IIHMSP '08 International Conference on 15-17 Aug. 2008
Page(s):1416 - 1421.

\bibitem{IEEEhowto:3}
Heewan Park; Hyun-il Lim; Seokwoo Choi; Taisook Han. \emph {A
Static Java Birthmark Based on Operand Stack Behaviors.}\hskip 1em
plus 0.5em minus 0.4em\relax Information Security and Assurance,
2008. ISA 2008. International Conference on 24-26 April 2008
Page(s):133 - 136.

\bibitem{IEEEhowto:4}
Devanbu, P.T.; Stubblebine, S.G. \emph {Stack and queue integrity
on hostile platforms.}\hskip 1em plus 0.5em minus 0.4em\relax
Software Engineering, IEEE Transactions on Volume 28,  Issue 1,
Jan. 2002 Page(s):100 - 108.

\bibitem{IEEEhowto:5}
Invisible Things Lab. \emph{BluePill Project.}\\
\hskip 1em plus 0.5em minus 0.4em\relax
http://www.bluepillproject.org/stuff/nbp-0.32-public.zip

\bibitem{IEEEhowto:6}
Xiaoxin Chen; Tal Garfinkel; E. Christopher Lewis; Pratap
Subrahmanyam; Carl A. Waldspurger;  Dan Boneh; Jeffrey Dwoskin;
Dan R.K. Ports. \emph {Overshadow: A Virtualization-Based Approach
to Retrofitting Protection in Commodity Operating Systems.}\hskip
1em plus 0.5em minus 0.4em\relax In ASPLOS'08, March 2008.

\bibitem{IEEEhowto:7}
Nick L. Petroni; PMichael Hicks. \emph {Automated detection of
persistent kernel control-flow attacks.}\hskip 1em plus 0.5em
minus 0.4em\relax Proceedings of the 14th Computer and
communications security, Oct. 2007, Alexandria, Virginia, USA.

\bibitem{IEEEhowto:8}
Invisible Things Lab. \emph {Is GameOver() Anyone?}\\
\hskip 1em plus 0.5em minus 0.4em\relax In Black Hat USA 2007, May
2007.

\bibitem{IEEEhowto:9}
E. Bugnion; S. Devine; M. Rosenblum. \emph {Disco: Running
Commodity Operating Systems on Scalable Multiprocessors.}\hskip
1em plus 0.5em minus 0.4em\relax In Proceedings of the Sixteenth
ACM Symposium on Operating Systems Principles, pages 143-156,
October 1997.

\bibitem{IEEEhowto:10}
Carl A. Waldspurger. \emph {Memory resource management in VMware
ESX server.}\hskip 1em plus 0.5em minus 0.4em\relax In Proceedings
of the Fifth Symposium on Operating Systems Design and
Implementation, pages 181-194, December 2002.

\bibitem{IEEEhowto:11}
M. Nelson; B.H. Lim; G. Hutchins. \emph {Fast Transparent
Migration for Virtual Machines.}\hskip 1em plus 0.5em minus
0.4em\relax In Proceedings of the USENIX Annual Technical
Conference, pages 391-394, April 2005.

\bibitem{IEEEhowto:12}
R. Ta-Min; L. Litty; D. Lie. \emph {Splitting Interfaces: Making
Trust Between Applications and Operating Systems
Configurable.}\hskip 1em plus 0.5em minus 0.4em\relax In
Proceedings of the Seventh Symposium on Operating Systems Design
and Implementation, pages 279-292, November 2006.

\bibitem{IEEEhowto:13}
PLin Tan; Ellick M. Chan; PReza Farivar; Nevedita Mallick; Jeffrey
C. Carlyle; Francis M. David; Roy H. Campbell. \emph {iKernel:
Isolating Buggy and Malicious Device Drivers Using Hardware
Virtualization Support.}\hskip 1em plus 0.5em minus 0.4em\relax In
Proceedings of the Third IEEE International Symposium on
Dependable, Autonomic and Secure Computing, September 2007.

\bibitem{IEEEhowto:15}
Igor Burdonov; Alexander Kosachev; Pavel Iakovenko. \emph
{Virtualization-based separation of privilege: working with
sensitive data in untrusted environment.}\hskip 1em plus 0.5em
minus 0.4em\relax In Proceedings of the 1st EuroSys Workshop on
Virtualization Technology for Dependable Systems, Mar. 2009.

\bibitem{IEEEhowto:16}
Artem Dinaburg; Paul Royal; Monirul Sharif; Wenke Lee. \emph
{Ether: malware analysis via hardware virtualization
extensions.}\hskip 1em plus 0.5em minus 0.4em\relax In Proceedings
of the 15th ACM conference on Computer and communications
security, Oct. 2008.

\bibitem{IEEEhowto:17}
Intel. \emph {Intel Trusted Execution Technology Architecture
Overview.}\hskip 1em plus 0.5em minus 0.4em\relax September 2006.

\bibitem{IEEEhowto:18}
Payne, B.D.; Carbone, M.; Sharif, M.; Wenke Lee; \emph {Lares: An
Architecture for Secure Active Monitoring Using
Virtualization}\hskip 1em plus 0.5em minus 0.4em\relax Security
and Privacy, 2008. SP 2008. IEEE Symposium on 18-22 May 2008
Page(s):233 - 247.

\bibitem{IEEEhowto:20}
Jun Yang; Lan Gao; Youtao Zhang. \emph{Improving Memory Encryption
Performance in Secure Processors.}\hskip 1em plus 0.5em minus
0.4em\relax IEEE Trans. Computers (TC) 54(5):630-640 (2005).

\bibitem{IEEEhowto:21}
Russinovich M.E.; Solomon, D.A. \emph{Microsoft Windows Internals,
Fourth Edition: Microsoft Windows Server 2003, Windows XP, and
Windows 2000.}\hskip 1em plus 0.5em minus 0.4em\relax Microsoft
Press (2005).

\end{thebibliography}


% that's all folks
\end{document}
